\documentclass{article}

\usepackage[margin=1.2in]{geometry}
\usepackage[colorlinks=true]{hyperref}
\usepackage{mathtools}
\usepackage{amsfonts}

\usepackage{tikz}

\usetikzlibrary{lindenmayersystems}

\pgfdeclarelindenmayersystem{Sierpinski triangle}{
    \symbol{X}{\pgflsystemdrawforward}
    \symbol{Y}{\pgflsystemdrawforward}
    \rule{X -> X-Y+X+Y-X}
    \rule{Y -> YY}
}%


\begin{document}

\section*{Overview}

Complex patterns often arise from repeatedly applying a simpler set of rules.
In this activity, we explore how two sets of rules generate a well-known fractal.\\

\noindent Work together as a group to complete the activities. Write the
answers to the questions on the large sheet of paper, and be sure to write
large enough  so that other groups can see your answers.

\section*{Activity A (18 minutes)}

\begin{enumerate}
	\item Grab a single sheet labeled ``Rule 90''. Tape it to a window or pin
	it to a board, so that all members in your group can see and draw on it.

	\item Beginning with the first row, fill out the first $6$ rows by applying
	the cellular automaton's rules. Make sure everyone understands how to
	apply the rules.

	\item Using a ``divide and conquer'' approach, quickly fill out the rest
	of the grid. This means that every one in the group should pick up a marker
	and try to apply the rules at the same time.

		\begin{itemize}
		\item[\textbf{Q1}:] What feature of the rules makes it possible to cooperatively
		fill out the rest of the grid?

		\item[\textbf{Q2}:] Could one person work on filling out the top half of the grid,
		while another person worked on filling out the bottom half?
		Why or why not?

		\item[\textbf{Q3}:] Why is color of the middle square in each triplet
	%	\tikz[scale=.3]{
		 %\def\i{0}
		 %\draw[fill=white] (\i + 0, 0) rectangle (\i + 1, 1);
     %\draw[fill=white] (\i + 1, 0) rectangle (\i + 2, 1);
     %\draw[fill=white] (\i + 2, 0) rectangle (\i + 3, 1);
  %}
  is not important? Find a simpler description of the rules that does not rely on
  the middle square. Describe it in words.

		\end{itemize}

\end{enumerate}

\section*{Activity B (12 minutes)}

\begin{enumerate}
	\item Grab a calculator and a single sheet labeled ``Pascal's Triangle''.
	Tape the sheet to a window or pin it to a board, so that all members in your
	group can see and draw on it.

	\item Begin at the top of the triangle and fill out the first 8 rows of
	Pascal's triangle, using the addition rule specified on the sheet. Make sure
	everyone understands how to apply the rule.

	\item Color all the circles with an even number inside them.

	\begin{itemize}
		\item[\textbf{Q1}:] Using only the colors of the circles, find a
		new set of rules that will allow you to finish coloring the rest of
		Pascal's triangle, without having to do addition.

		\item [\textbf{Q2}:] How does the rule on colors relate to the properties
		of even and oddness?

	\end{itemize}

	\item Finish coloring Pascal's triangle. [It is not necessary to put numbers
	inside the circles.]

	\begin{itemize}

		\item[\textbf{Q3}:] What pattern do you see? Compare and contrast it
		to the pattern we saw for ``Rule 90''.


		\item[\textbf{Q4}:] How does the coloring rule set compare to the simplified
		rule set that we found for ``Rule 90''?

	\end{itemize}


\end{enumerate}

\newpage

\section*{Bonus Questions}
	\begin{enumerate}
		\item Where does the ``$90$'' in ``Rule $90$'' come from? Hint: $90 = 2^6 + 2^4 + 2^3 + 2^1$.

		\item The Sierpinski triangle can also be generated using the following rule:
		\def\trianglewidth{.5cm}%
		\def\level{0}
		\tikzset{
		    l-system={step=\trianglewidth/(2^\level), order=\level, angle=-120}
		}%
		\tikz[baseline=1pt]{
		    \fill [black] (0,0) -- ++(0:\trianglewidth) -- ++(120:\trianglewidth) -- cycle;
		    \draw [draw=none] (0,0) l-system
		    [l-system={Sierpinski triangle, axiom=X},fill=white];
		}
		{\Large$\rightarrow$}
		\def\level{1}
		\tikzset{
		    l-system={step=\trianglewidth/(2^\level), order=\level, angle=-120}
		}%
		\tikz[baseline=1pt]{
		    \fill [black] (0,0) -- ++(0:\trianglewidth) -- ++(120:\trianglewidth) -- cycle;
		    \draw [draw=none] (0,0) l-system
		    [l-system={Sierpinski triangle, axiom=X},fill=white];
		}


		Begin with an equilateral triangle, having side length equal to $1$.
		Then, apply the rule repeatedly. What are the formulas for the perimeter
		and area of the pattern as a function of the number of iterations? What is
		the perimeter and area as the number of iterations tends to infinity?\\

		\item For Pascal's triangle, $x$ and $y$ correspond to $\binom{n}{k}$ and
				  $\binom{n}{k+1}$. The ``$x+y$'' circle corresponds to $\binom{n+1}{k+1}$.
		      Show that generation rule takes the following form:
		      $\binom{n}{k} + \binom{n}{k+1} = \binom{n+1}{k+1}$.\\

	\end{enumerate}

\section*{Resources}

If you want to learn more, try searching for some of the following keywords:
\begin{quote}
		fractals, cellular automata, iterated function systems, chaos game, \\Sierpinski triangle, Turing complete, game of life, Mandelbrot set
\end{quote}

\noindent Here are some fun webpages and a nice YouTube video:
\begin{quote}
	\url{http://ecademy.agnesscott.edu/~lriddle/ifs/ifs.htm}\\
	\url{http://www.kevs3d.co.uk/dev/lsystems}\\
	\url{https://www.youtube.com/watch?v=5plLxMnbtAw}
\end{quote}

\end{document}
